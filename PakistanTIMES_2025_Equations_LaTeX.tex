\documentclass[12pt,a4paper]{article}
\usepackage[utf8]{inputenc}
\usepackage[T1]{fontenc}
\usepackage{geometry}
\usepackage{amsmath}
\usepackage{amsfonts}
\usepackage{amssymb}
\usepackage{graphicx}
\usepackage{booktabs}
\usepackage{longtable}
\usepackage{array}
\usepackage{multirow}
\usepackage{wrapfig}
\usepackage{float}
\usepackage{colortbl}
\usepackage{pdflscape}
\usepackage{tabu}
\usepackage{threeparttable}
\usepackage{threeparttablex}
\usepackage{forloop}
\usepackage{booktabs}
\usepackage{enumitem}
\usepackage{hyperref}
\usepackage{mathtools}
\usepackage{physics}

\geometry{margin=1in}

\title{\textbf{PakistanTIMES 2025: Mathematical Formulations and Equations}\\
\large Complete Mathematical Framework for Energy System Modeling}

\author{Energy Systems Research Group\\
Pakistan Institute of Energy Studies}

\date{\today}

\begin{document}

\maketitle

\begin{abstract}
This document provides a comprehensive mathematical formulation of the PakistanTIMES 2025 energy system model. It includes all equations for demand forecasting, capacity planning, investment calculations, emissions modeling, and system constraints. The formulations are based on the TIMES (The Integrated MARKAL-EFOM System) framework and adapted for Pakistan's specific energy system characteristics and constraints.
\end{abstract}

\tableofcontents
\newpage

\section{Introduction}
The PakistanTIMES 2025 model uses a comprehensive mathematical framework to optimize Pakistan's energy system evolution from 2014 to 2050. This document presents all mathematical formulations, equations, and constraints used in the model.

\section{Demand Forecasting Equations}

\subsection{Basic Demand Growth Model}
The electricity demand is modeled using a compound growth model with GDP and population drivers:

\begin{equation}
\label{eq:basic_demand}
D(t) = D_0 \times (1 + g)^t
\end{equation}

Where:
\begin{itemize}
    \item $D(t)$ = Electricity demand in year $t$ (TWh)
    \item $D_0$ = Base year demand (2014) = 87.34 TWh
    \item $g$ = Annual growth rate
    \item $t$ = Years from base year
\end{itemize}

\subsection{GDP-Driven Demand Model}
Demand is also modeled as a function of GDP growth and electricity intensity:

\begin{equation}
\label{eq:gdp_demand}
D(t) = \text{GDP}(t) \times \text{EI}(t) \times \eta(t)
\end{equation}

Where:
\begin{itemize}
    \item $\text{GDP}(t)$ = Gross Domestic Product in year $t$ (Billion USD)
    \item $\text{EI}(t)$ = Electricity intensity (kWh/USD)
    \item $\eta(t)$ = Electrification factor (0 to 1)
\end{itemize}

\subsection{GDP Growth Projection}
\begin{equation}
\label{eq:gdp_growth}
\text{GDP}(t) = \text{GDP}_0 \times (1 + g_{\text{GDP}})^t
\end{equation}

Where:
\begin{itemize}
    \item $\text{GDP}_0$ = Base year GDP (2014) = 244.4 Billion USD
    \item $g_{\text{GDP}}$ = Annual GDP growth rate = 4.5-5.5\%
\end{itemize}

\subsection{Population Growth Projection}
\begin{equation}
\label{eq:pop_growth}
P(t) = P_0 \times (1 + g_{\text{pop}})^t
\end{equation}

Where:
\begin{itemize}
    \item $P_0$ = Base year population (2014) = 185.0 million
    \item $g_{\text{pop}}$ = Annual population growth rate = 1.2-2.0\%
\end{itemize}

\subsection{Per Capita Electricity Demand}
\begin{equation}
\label{eq:per_capita}
D_{\text{pc}}(t) = \frac{D(t)}{P(t)}
\end{equation}

Where:
\begin{itemize}
    \item $D_{\text{pc}}(t)$ = Per capita electricity demand (kWh/person)
\end{itemize}

\section{Capacity Planning Equations}

\subsection{Peak Demand Calculation}
Peak demand is calculated using load factor:

\begin{equation}
\label{eq:peak_demand}
P_{\text{peak}}(t) = \frac{D(t)}{\text{LF} \times 8760}
\end{equation}

Where:
\begin{itemize}
    \item $P_{\text{peak}}(t)$ = Peak demand in year $t$ (GW)
    \item $\text{LF}$ = Load factor = 0.65 (typical for Pakistan)
    \item 8760 = Hours per year
\end{itemize}

\subsection{Required Capacity Calculation}
Total required capacity includes reserve margin:

\begin{equation}
\label{eq:required_capacity}
C_{\text{req}}(t) = P_{\text{peak}}(t) \times (1 + \text{RM}(t))
\end{equation}

Where:
\begin{itemize}
    \item $C_{\text{req}}(t)$ = Required capacity in year $t$ (GW)
    \item $\text{RM}(t)$ = Reserve margin in year $t$ (20\% $\rightarrow$ 12\%)
\end{itemize}

\subsection{Reserve Margin Evolution}
\begin{equation}
\label{eq:reserve_margin}
\text{RM}(t) = \text{RM}_0 - (\text{RM}_0 - \text{RM}_{\text{final}}) \times \frac{t}{T_{\text{total}}}
\end{equation}

Where:
\begin{itemize}
    \item $\text{RM}_0$ = Initial reserve margin = 20\%
    \item $\text{RM}_{\text{final}}$ = Final reserve margin = 12\%
    \item $T_{\text{total}}$ = Total time period = 36 years
\end{itemize}

\section{Technology Mix Equations}

\subsection{Renewable Share Constraint}
Renewable share follows target trajectory:

\begin{equation}
\label{eq:renewable_share}
R_{\text{share}}(t) = R_0 + (R_{\text{target}} - R_0) \times \frac{t}{T_{\text{total}}}
\end{equation}

Where:
\begin{itemize}
    \item $R_{\text{share}}(t)$ = Renewable share in year $t$
    \item $R_0$ = Initial renewable share = 5\%
    \item $R_{\text{target}}$ = Target renewable share (30\%, 50\%, 60\%, 70\%)
\end{itemize}

\subsection{Renewable Capacity Requirement}
\begin{equation}
\label{eq:renewable_capacity}
C_{\text{ren}}(t) = C_{\text{req}}(t) \times R_{\text{share}}(t)
\end{equation}

Where:
\begin{itemize}
    \item $C_{\text{ren}}(t)$ = Renewable capacity in year $t$ (GW)
\end{itemize}

\subsection{Thermal Capacity Requirement}
\begin{equation}
\label{eq:thermal_capacity}
C_{\text{thermal}}(t) = C_{\text{req}}(t) \times (1 - R_{\text{share}}(t))
\end{equation}

Where:
\begin{itemize}
    \item $C_{\text{thermal}}(t)$ = Thermal capacity in year $t$ (GW)
\end{itemize}

\section{Investment Calculations}

\subsection{Capacity Addition Investment}
Investment for new capacity additions:

\begin{equation}
\label{eq:capacity_investment}
I_{\text{cap}}(t) = \sum_i (C_{\text{add},i}(t) \times \text{UC}_i(t))
\end{equation}

Where:
\begin{itemize}
    \item $I_{\text{cap}}(t)$ = Capacity investment in year $t$ (Billion USD)
    \item $C_{\text{add},i}(t)$ = Capacity addition for technology $i$ in year $t$ (GW)
    \item $\text{UC}_i(t)$ = Unit cost for technology $i$ in year $t$ (USD/kW)
\end{itemize}

\subsection{Unit Cost Evolution with Learning}
\begin{equation}
\label{eq:unit_cost}
\text{UC}_i(t) = \text{UC}_{0,i} \times \left(\frac{C_{\text{cum},i}(t)}{C_{0,i}}\right)^{-\text{LR}_i}
\end{equation}

Where:
\begin{itemize}
    \item $\text{UC}_{0,i}$ = Initial unit cost for technology $i$
    \item $C_{\text{cum},i}(t)$ = Cumulative capacity for technology $i$
    \item $C_{0,i}$ = Initial cumulative capacity
    \item $\text{LR}_i$ = Learning rate for technology $i$
\end{itemize}

\subsection{Operations and Maintenance Costs}
\begin{equation}
\label{eq:om_costs}
I_{\text{OM}}(t) = \sum_i (C_i(t) \times \text{OM}_i(t))
\end{equation}

Where:
\begin{itemize}
    \item $I_{\text{OM}}(t)$ = O\&M costs in year $t$ (Billion USD)
    \item $C_i(t)$ = Installed capacity for technology $i$ (GW)
    \item $\text{OM}_i(t)$ = O\&M cost rate for technology $i$ (USD/kW/year)
\end{itemize}

\section{Emissions Modeling}

\subsection{Annual Emissions Calculation}
Annual CO$_2$ emissions from electricity generation:

\begin{equation}
\label{eq:annual_emissions}
E(t) = \sum_i (G_i(t) \times \text{EF}_i(t))
\end{equation}

Where:
\begin{itemize}
    \item $E(t)$ = Annual emissions in year $t$ (MtCO$_2$)
    \item $G_i(t)$ = Generation from technology $i$ in year $t$ (TWh)
    \item $\text{EF}_i(t)$ = Emission factor for technology $i$ (tCO$_2$/MWh)
\end{itemize}

\subsection{Generation Calculation}
\begin{equation}
\label{eq:generation}
G_i(t) = \frac{C_i(t) \times \text{CF}_i(t) \times 8760}{1000}
\end{equation}

Where:
\begin{itemize}
    \item $\text{CF}_i(t)$ = Capacity factor for technology $i$
    \item 8760 = Hours per year
    \item 1000 = Conversion from MWh to TWh
\end{itemize}

\subsection{Cumulative Emissions}
\begin{equation}
\label{eq:cumulative_emissions}
E_{\text{cum}}(t) = \sum_{i=2014}^t E(i)
\end{equation}

Where:
\begin{itemize}
    \item $E_{\text{cum}}(t)$ = Cumulative emissions from 2014 to year $t$ (GtCO$_2$)
\end{itemize}

\section{System Constraints}

\subsection{Energy Balance Constraint}
Total generation must equal demand plus losses:

\begin{equation}
\label{eq:energy_balance}
\sum_i G_i(t) = D(t) \times (1 + L(t))
\end{equation}

Where:
\begin{itemize}
    \item $L(t)$ = System losses in year $t$ (typically 15-20\%)
\end{itemize}

\subsection{Capacity Adequacy Constraint}
Installed capacity must meet peak demand plus reserve:

\begin{equation}
\label{eq:capacity_adequacy}
\sum_i C_i(t) \geq P_{\text{peak}}(t) \times (1 + \text{RM}(t))
\end{equation}

\subsection{Technology Availability Constraint}
Technology deployment limited by resource availability:

\begin{equation}
\label{eq:technology_availability}
C_i(t) \leq C_{\text{max},i}(t)
\end{equation}

Where:
\begin{itemize}
    \item $C_{\text{max},i}(t)$ = Maximum available capacity for technology $i$
\end{itemize}

\section{Optimization Objective}

\subsection{Total System Cost Minimization}
The model minimizes total system cost:

\begin{equation}
\label{eq:objective_function}
\text{Minimize: } \sum_t (I_{\text{cap}}(t) + I_{\text{OM}}(t) + I_{\text{fuel}}(t)) \times \text{DF}(t)
\end{equation}

Where:
\begin{itemize}
    \item $I_{\text{fuel}}(t)$ = Fuel costs in year $t$
    \item $\text{DF}(t)$ = Discount factor for year $t$
\end{itemize}

\subsection{Discount Factor}
\begin{equation}
\label{eq:discount_factor}
\text{DF}(t) = \frac{1}{(1 + r)^t}
\end{equation}

Where:
\begin{itemize}
    \item $r$ = Discount rate = 8\% (typical for energy projects)
\end{itemize}

\section{Validation and Calibration Equations}

\subsection{Historical Calibration}
Model calibration against historical data:

\begin{equation}
\label{eq:calibration}
\text{Minimize: } \sum_t \frac{(D_{\text{model}}(t) - D_{\text{hist}}(t))^2}{D_{\text{hist}}(t)^2}
\end{equation}

Where:
\begin{itemize}
    \item $D_{\text{model}}(t)$ = Modeled demand in year $t$
    \item $D_{\text{hist}}(t)$ = Historical demand in year $t$
\end{itemize}

\subsection{Peer Literature Validation}
Validation against peer-reviewed studies:

\begin{equation}
\label{eq:validation}
V = \frac{|D_{\text{model}}(2050) - D_{\text{literature}}(2050)|}{D_{\text{literature}}(2050)}
\end{equation}

Where:
\begin{itemize}
    \item $V$ = Validation metric (target: $V < 0.2$ or 20\%)
    \item $D_{\text{literature}}(2050)$ = Literature projection for 2050
\end{itemize}

\section{Sensitivity Analysis Equations}

\subsection{Parameter Sensitivity}
Sensitivity of results to parameter changes:

\begin{equation}
\label{eq:sensitivity}
S_{ij} = \frac{\partial Y_i}{\partial X_j} \times \frac{X_j}{Y_i}
\end{equation}

Where:
\begin{itemize}
    \item $S_{ij}$ = Sensitivity of output $i$ to parameter $j$
    \item $Y_i$ = Output variable $i$
    \item $X_j$ = Input parameter $j$
\end{itemize}

\subsection{Monte Carlo Sampling}
\begin{equation}
\label{eq:monte_carlo}
X_{j,k} = X_{j,\text{base}} + \varepsilon_{j,k} \times \sigma_j
\end{equation}

Where:
\begin{itemize}
    \item $X_{j,k}$ = Parameter $j$ in sample $k$
    \item $X_{j,\text{base}}$ = Base value of parameter $j$
    \item $\varepsilon_{j,k}$ = Random error for parameter $j$ in sample $k$
    \item $\sigma_j$ = Standard deviation of parameter $j$
\end{itemize}

\section{References}
\begin{enumerate}
    \item Loulou, R., et al. (2005). The TIMES Model and its Applications. Energy Economics.
    \item IEA (2023). World Energy Outlook 2023. International Energy Agency.
    \item Pakistan Ministry of Energy (2023). Pakistan Energy Yearbook 2023.
    \item UNFCCC (2021). Pakistan's Nationally Determined Contribution.
    \item World Bank (2023). Pakistan Energy Sector Assessment.
    \item Markal, A. (1976). A Linear Programming Model for Energy-Economy Analysis. Brookhaven National Laboratory.
    \item EFOM (1983). Energy Flow Optimization Model. Commission of the European Communities.
\end{enumerate}

\section{Appendices}

\subsection{Appendix A: Parameter Values}
Complete list of all parameter values used in the model.

\subsection{Appendix B: Technology Data}
Detailed technology characteristics, costs, and performance data.

\subsection{Appendix C: Scenario Definitions}
Complete definition of all 16 scenarios analyzed.

\end{document}
